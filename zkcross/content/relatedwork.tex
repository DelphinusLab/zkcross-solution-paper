\section{Related Works}
\label{chp:related-work}
\noindent
\emph{Notary Schemes.} The main idea of notary schemes is to elect one or more trusted nodes as a notary public and report transactions in different blockchain networks through notaries \cite{qin2018overview}. Therefore, all information transferred between different blockchains is completely managed by notaries. The centralized notary scheme has efficiency in procession events and simplicity in implementation. However, it suffers from the SPOF problem. Therefore, a multi-signature notary scheme may be proposed to reduce the trust in a certain centralized node. However, this multi-signature notary is surely non-permissionless and needs extra protocols to ensure liveness.

\smallskip\noindent\emph{SPV: Simplified Payment Verification.}
SPV \cite{lin2017survey,ray2020blwn} is a special case of one-directional state pining \cite{robinson2020merits}. In such protocols, the block header of $C_A$ and a Merkle proof of a particular transaction is monitored and sent to the target blockchain $C_B$. $C_B$ accepts the transaction by calculating the partial Merkle tree hash and compares it with the block header of $C_A$. The liveness property of this approach depends on the liveness of relayers of $C_A$ and the safety property relies on the block headers being reported to $C_B$ honestly. 
    
\smallskip\noindent\emph{Sidechains.}
The goal of sidechains \cite{singh2020sidechain} is to extend the scalability and functionality of the blockchain system. Sidechains can enforce the security of transactions on themselves by implementing a protocol that can be validated by consensus. Since the sidechain needs to update state changes back to the underlying blockchains, the blockchains need to trust or verify the transactions sent out by sidechains. The safety property again relies on whether observation of the source chain can be honestly reported to the side chain and whether the verifier on the target chain can reject all fraud transactions. The liveness of sidechains relies on how robust the sidechain itself is and whether all the transactions that happened on the sidechain will get reported to the target chain eventually. Some improving ideas are given in Plasma \cite{poon2017plasma}.

\smallskip\noindent\emph{Cross Chain Gateway and Relayers.}
Cross Chain Gateway with relays is another extension of the idea of state pinning. While it enables blockchain interoperability applications including cross-chain token transfer, the safety property is attained at the cost of storing every single block header of the source blockchain \cite{belchior2021survey}. In general, the cost of storing such a state is very expensive.

\smallskip\noindent\emph{ZKBridge.}
A recent protocol, ZKBridge \cite{xie2022zkbridge-zkbridge}, proposed an efficient cross-chain bridge that guarantees strong security by using ZKP to provide trustless relayers. The trust base of ZKBridge requires that the light-client protocols of checking the blockhead of the source blockchain are secure. Compared with ZKBridge, our protocol is completely different in two aspects. First, we use a consensus algorithm to make sure the blockhead tracking process is permissionless. Second, ZKBridge still follows the traditional way of executing transactions on different blockchains and using relayers to synchronize them, so it does not scale well in the situation of a large number of involved blockchains since it needs to handle the interference between the blockchains in a case-by-case manner; instead, we simulate the transaction on our aggregator chain and convince native blockchains to update their local state by providing the ZKP of the simulation result, and hence provides a general solution to handle the interference. Thus, our protocol naturally has the linearizability property with no scalability problem when applied to scenarios with a large number of native blockchains.
%\newline
%In conclusion, we have the following (see Table: \ref{tab:related}).
%\begin{table*}[!h]
%\small
%\centering
%\caption{Related Work}\label{tab:related}
%\begin{tabular}{ | c | c | c | c | c |}
%\hline
%Protocol & Liveness & Trustless Safety & Permissionless & MultiChain Atomicity \\
%\hline
%NotarySchemes & yes & yes & no & no\\
%\hline
%SPV & yes & no & no & no \\
%\hline
%Peg & no & no & no & no\\
%\hline
%SideChain & yes & no & yes & yes\\ 
%\hline
%Gateway& yes & no & no & no\\ 
%\hline
%\end{tabular}
%\end{table*}

