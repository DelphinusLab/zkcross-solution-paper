\section{Implementation and Evaluation}
\label{chp:bench}
\subsection{Code base for \dprotocol}
We implemented the Aggregator chain based on substrate \cite{substrate}. Substrate is a blockchain development kit, with which users can customize a blockchain by combining rich blockchain building blocks. We start with a standard substrate setting and build our \dprotocol by providing our own consensus component, simulation component and finalizing component. As we described in Section \ref{consensus-stage}, the verifying process is done on native blockchains. Thus the main performance cost is spent on the ZKP part during consensus, simulation and proving stage. Since the ZKP performance is highly related to CS (constraints size) of ZKP circuits, below we will mainly use the number of constraints as a measurement of performance. 

\subsection {Consensus Component}
Recall that the consensus component needs to provide ZKP circuit for registering valid node, validating a registered node and voting the node to generate a new block. We uses the ECDSA signature on BabyJubJub \cite{whitehat2020baby} curve to reduce the circuit size. Also the voting process can be further splited into two parts: voter needs to generate and sending the voting ticket to target and winner needs to prove that he has received sufficient vote.  Thus the following four different circuits written in circom \cite{munoz2022circom} are needed for the consensus component (see Table \ref{tbl:consensus-cs}).

\begin{table}[!ht]
\small
\centering
\caption{Consensus Circuit Constraint Size}
\label{tbl:consensus-cs}
\begin{tabular}{ | c | c | c | c | c | }
\hline
Total Nodes & Circuit Name & Language & CS & Proofs \\
\hline
128 & register & circom & $2^{18}$ & 1\\
\hline
256 & register & circom & $2^{18}$ & 1\\
\hline
$2^{18}$ & register & circom & $2^{21}$ & 1\\
\hline
128 & node validation & circom & $2^{17}$ & 1\\
\hline
256 & node validation & circom & $2^{17}$ & 1\\
\hline
$2^{12}$ & node validation & circom & $2^{21}$ & 1 \\
\hline
128 & source voting & circom & $2^{17}$ & 1\\
\hline
256 & source voting & circom & $2^{17}$ & 1\\
\hline
$2^{12}$ & source voting & circom & $2^{17}$ & 1\\
\hline
128 & winner proving & circom & $2^{24}$ & 1\\
\hline
256 & winner proving & circom & $2^{24}$ & 2 \\
\hline
$2^{12}$ & winner proving & circom & $2^{24}$ & 16 \\
\hline
\end{tabular}
\end{table}
\begin{remark}
When winner proving that he has collected enough votes, he might need to provide more than one proof and each proves that certain amount of voter has voted him by sending him the voting proof.
\end{remark}

\subsection{Simulation Circuit}
Since the size of the simulation circuit usually depends on the transaction logic a user would like to execute, we provide an example to show the circuits size of building a multi-chain DEFI (decentralized finance \cite{zetzsche2020decentralized, chen2020blockchain}) system over ZK Multi-Blockchain Aggregator (see Table \ref{tbl:defi-cs}) using circom. In Table \ref{tbl:defi-cs} the AMM circuits denotes the multi-blockchain transactions of automated market maker \cite{mohan2022automated} and the state circuit denotes the circuit of Merkle tree manipulating interfaces. 

\begin{table}[h]
\small
\centering
\caption{DEFI Transaction Circuits}
\label{tbl:defi-cs}
\begin{tabular}{ | c | c | c | c | }
\hline
Number of $t_x$ & Circuit Name & Language & CS \\
\hline
8 & AMM circuit & Circom & $2^{14}$ \\
\hline
256 & AMM circuit & Circom & $2^{19}$ \\
\hline
8 & State circuit & Circom & $2^{16}$ \\
\hline
256 & State circuit & Circom & $2^{21}$ \\
\hline
\end{tabular}
\end{table}

\subsection{Proof Batching 
circuit}
The common way to batch multiple proofs into one single proof is to encode the PLONK verify algorithm into ZKP circuits. Since the verify algorithm contains a lot of calculation of EC (elliptic curve) scalar multiplication, it is infeasible to implement such complex algorithm in Circom. Thus we encodes the algorithm using Halo2 \cite{halo2book} proving system (see Figure \ref{tbl:batch-proofs} for constraint size in Halo2).  

\begin{table}[!ht]
\small
\centering
\caption{Circuits for Batching Proofs}
\label{tbl:batch-proofs}
\begin{tabular}{ | c | c | c | c | }
\hline
Circuit Name & Proof System & CS \\
\hline
EC scalar multiplication circuit & Halo2 & $7 \times 10^4$ \\
\hline
Transcript circuit & Halo2 & $2^{18}$ \\
\hline
Batch circuit & Halo2 & $2^{24}$ \\
\hline
\end{tabular}
\end{table}

\subsection{Evaluation Results on Real-world Testnet}
The overall block generating time is the sum (some of the calculation can be executed in a concurrent manner) of the consensus stage, simulating stage, and proving stage. The time consuming of simulating stage is negligible, thus it is sufficient to calculate the time of consensus stage and proving stage. The average time consuming of $2^{20}$ constraints is about 6 sec using Rapidsnark on a 2 (Intel(R) Core(TM) i7) CPU computer with 128G memory. When batch proofs using Halo2 system, it takes 34 seconds more to batch the consensus proof with around 32 simulation proofs in a block on a 2 (Intel(R) Core(TM) i7) CPU computer with one NVIDIA GeForce RTX 3070 card.

When running the whole DEFI application on \dprotocol testnet over four native blockchains (goerli, bsctestnet, cronostestnet, rolluxtestnet), the overall performance of transaction handling is about 1 TPS (transaction per second) per machine (2 Intel i7 CPU + GeForce RTX 3070 graphic card) and can be scaled linearly if more hardware is provided.

